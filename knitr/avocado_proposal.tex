% DAR 3 proposal: predicting expenditures in NY municipalities


\documentclass[11pt]{article}

\usepackage{graphics}
\usepackage{latexsym}
\usepackage{color}
\usepackage{epstopdf} % required by TeXShop and Mac to process eps figures in includegraphics environment

% Approximately 1 inch borders all around
\setlength\topmargin{-.56in}
\setlength\evensidemargin{0in}
\setlength\oddsidemargin{0in}
\setlength\textwidth{6.49in}
\setlength\textheight{8.6in}

\let\BLS=\baselinestretch

% macros for different spacing options
\makeatletter
\newcommand{\singlespacing}{\let\CS=\@currsize\renewcommand{\baselinestretch}{1}\small\CS}
\newcommand{\doublespacing}{\let\CS=\@currsize\renewcommand{
\baselinestretch}{1.5}\small\CS}
\newcommand{\normalspacing}{\let\CS=\@currsize\renewcommand{\baselinestretch}{\BLS}\small\CS}
\makeatother

% macros I commonly use, primarily for boldfaced greek letters
\def\R{\mbox{\rlap{I}\hskip .03in R}}
\newcommand{\bfbeta}{\mbox{\boldmath $\beta$}}
\newcommand{\bfeps}{\mbox{\boldmath $\epsilon$}}
\newcommand{\bfsigma}{\mbox{\boldmath $\sigma$}}
\newcommand{\bfpsi}{\mbox{\boldmath $\Psi$}}
\newcommand{\bft}{\mbox{\boldmath $\theta$}}
\newcommand{\bfmu}{\mbox{\boldmath $\mu$}}
%%%%%%
% To make comments, in red, in the margin of text
\newcommand{\but}[1]{\marginpar{\color{red}\begin{sffamily}#1\end{sffamily}\color{black}}}
% Example of usage: \but{I deleted a sentence - see the latex file}  
%%%%%%


\begin{document}

\title{Proposal: Avocado Price}
\author{Kristine Dinh\thanks{Email: kdinh@sdsu.edu} \\
San Diego State University \\ STAT 794: Statistical Communication in Data Science}

\date{\today}

\maketitle
%\setlength{\parskip}{0in}
%\begin{abstract}
%\setlength{\parindent}{0in}
%
%Executive summary here.
%
%\end{abstract}

\noindent {\em Background}

We are interested in predicting the prices of avocados in different regions. Avocado lovers wish to purchases good avocados with lower prices to make avocado toast for breakfast or guacamole to dip with chips. However, there are no way of know which avocado has good price and where should ones buy avocado at. 

\vspace{0.5cm}
\noindent {\em Specific Aims}

\begin{itemize}
\item Study the relationship between avocado characteristics, geography and average price. 
\item Predict average avocado price in 2020 and compare the predicted and empirical avocado prices to access the quality of the model.
\item Determine if the model is good to forecast average avocado price for next year, 2021. 
\end{itemize}

\vspace{0.5cm}
\noindent {\em Data}

This data set was obtained from Kaggle. It is originally from the Hass Avocado Broad (HAB) website. This dataset has historical data of avocado prices and characteristics. The dataset contains two time series columns, and 10 predictors, as shown in table below. PLU stands for Price Look-Up code. Other variables are self-explanatory. 
\begin{center}
\begin{tabular}{|r|l|}
\hline
Variable & Definition \\ \hline
Date & The date of the avocado observed \\
Year & Year of observation\\ \hline
Type & Conventional or organic \\ 
Region & The city or region of the avocado \\
Total Volume & Total number of avocados sold \\
4046 & Total number of avocados with PLU 4046 sold\\
4225 & Total number of avocados with PLU 4225 sold \\
4770 & Total number of avocados with PLU 4770 sold \\
Total Bags & Total number of bags sold\\
Small Bags & Total number of small bags sold\\
Large Bags & Total number of large bags sold\\
X-large Bags & Total number of  extra large bags sold\\ \hline
\end{tabular}
\end{center}

The data is from 2015 to 2020 with 54 distinct geographical regions of where the avocados are from. Avocado lovers want to build a model that can predict the price of avocado.The train set of the model will be using avocado observed between 2015 and 2019. All predictors of avocado prices in 2020 are provided in the dataset along with the price of avocado to be the test set. If the model can generate predicted price similar to empirical avocado price, avocado lovers will be more likely to use this model to predict the price of avocado for future years. With the predicted avocado prices, individuals who eat avocado toast and guacamole would have a better understanding of what kind of avocado and where the avocado come from would decrease or increases the price. 


\vspace{0.5cm}
\noindent {\em Methods}

To serve the purpose of this data analysis, we will build the model using a multiple linear regression with 11 possible predictors. The analysis will start with a data exploratory analysis to determine the relationship between avocado prices and each of the independent variable. After the exploratory analysis process, we will develop the model using the significant variables. During this process, we want to select the best variables for our model using different criterion including AIC, BIC, MSE, and R-square. After selecting the best model, we would check for the model quality by running diagnostics on the model to assess the normality of the residuals, influential points, multicollinearity, etc..

\vspace{0.5cm}
\noindent {\em Broader Impacts}

This multiple linear regression model will be the main tool to point out the relationship between average avocado prices and other avocado characteristics and locations. We will predict the prices of avocado in 2020 and compare the predicted price to the empirical price. If the errors of the predicted and empirical avocado prices are small, we can use this model to project the prices of next year. Finally, we will talk about the limitation of this dataset and suggest future ideas on this dataset to improve and implement the model.  Of course, all report on this project will be shared with details documentations and reproducible R codes.

\end{document}
