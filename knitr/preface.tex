% Preface to a LaTeX document for Stat 696: margins, spacing, packages, macros,
% title, abstract, and figures.

\documentclass[11pt]{article}

\usepackage{graphics}
\usepackage{latexsym}
\usepackage{color}
\usepackage{epstopdf} % required by TeXShop and Mac to process eps figures in includegraphics environment

% Approximately 1 inch borders all around
\setlength\topmargin{-.56in}
\setlength\evensidemargin{0in}
\setlength\oddsidemargin{0in}
\setlength\textwidth{6.49in}
\setlength\textheight{8.6in}

\let\BLS=\baselinestretch

% macros for different spacing options
\makeatletter
\newcommand{\singlespacing}{\let\CS=\@currsize\renewcommand{\baselinestretch}{1}\small\CS}
\newcommand{\doublespacing}{\let\CS=\@currsize\renewcommand{
\baselinestretch}{1.5}\small\CS}
\newcommand{\normalspacing}{\let\CS=\@currsize\renewcommand{\baselinestretch}{\BLS}\small\CS}
\makeatother

% macros I commonly use, primarily for boldfaced greek letters
\def\R{\mbox{\rlap{I}\hskip .03in R}}
\newcommand{\bfbeta}{\mbox{\boldmath $\beta$}}
\newcommand{\bfeps}{\mbox{\boldmath $\epsilon$}}
\newcommand{\bfsigma}{\mbox{\boldmath $\sigma$}}
\newcommand{\bfpsi}{\mbox{\boldmath $\Psi$}}
\newcommand{\bft}{\mbox{\boldmath $\theta$}}
\newcommand{\bfmu}{\mbox{\boldmath $\mu$}}
%%%%%%
% To make comments, in red, in the margin of text
\newcommand{\but}[1]{\marginpar{\color{red}\begin{sffamily}#1\end{sffamily}\color{black}}}
% Example of usage: \but{I deleted a sentence - see the latex file}  
%%%%%%


\begin{document}

\title{Title here}
\author{Richard A. Levine\thanks{Email: rlevine@mail.sdsu.edu} \\
San Diego State University \\ STAT 794: Statistical Communication in Data Science}

\date{\today}
\maketitle
\setlength{\parskip}{0in}
\begin{abstract}
\setlength{\parindent}{0in}

Executive summary here.

\end{abstract}


%\newpage

\section{Introduction}
\label{intro}

% For mark-ups in revising, here is how to highlight text in colors; this example is a remark in blue
\textcolor{blue}{Make sure to compile your \LaTeX\ code multiple times to get all the citations and labels correct.}

\section{Methods}
\label{method}

Three primary ways to include math (and will show an itemized list at the same time): 

\begin{itemize} 

\item Place math inside the text using \$ \{math here\} \$, e.g., $\theta = x^2$.  

\item Display the equation using the {\tt displaymath} or {\tt equation} environment.  The {\tt displaymath} environment does not label the equation:
\[ \gamma = \alpha + \beta \cdot \ln(y).\]
The {\tt equation} environment allows you to label the equation:
\begin{equation}
\label{fraction}
\xi = \frac{\nu^{\epsilon}}{\sum_{i=1}^n s^2_i}.
\end{equation}
Equation (\ref{fraction}) is labeled ``fraction'' and may be cited by using {\tt ref}.

\item Finally, there is the equation array {\tt eqnarry*} which stacks equations nicely for you.  The asterisk tells \LaTeX\ not to label any parts of this equation.
\begin{eqnarray*}
m({\bf x}, \Delta | {\bf z}, T) 
	& = & \int \prod_{i=1}^b  
	 f({\bf x}_i, \Delta_i | \bfbeta_i, \gamma_i, T) \\
	& & \hspace{1cm} \times \pi(\bfbeta_i|T) 
	\pi(\gamma_i|T) \, d\bfbeta_i d\gamma_i \\
	& \approx & \exp\{{\cal L}(\lambda^*_i, \gamma^*_i)\} (2\pi)^{p/2}|-
	 {\cal L}''(\lambda^*_i, \gamma^*_i)|^{-1/2}.
\end{eqnarray*}
\end{itemize}
If an equation ends a sentence, please remember that you still need to place a period at the end of the equation.

\section{Results}
\label{results}

In this section we show a table using {\tt table}, {\tt tabular}, and {\tt center} environments.  Table~\ref{transcript} is labeled ``transcript" and like equations can be cited using {\tt ref}.  Just make sure to {\tt label} the table after the {\tt caption} to get the numbering right.

\begin{table}
\caption{My Fall 2016 transcript.}
\label{transcript}
% Label must come after caption to get numbering right
\begin{center}
\begin{tabular}{|r|c|l|} \hline
Class & Units & Grade \\ \hline
Stat 696 & 3 & A \\ \hline
Stat 700 & 3 & A- \\ \hline
\end{tabular}
\end{center}
\end{table}


\subsection{Free \LaTeX\ book}

Feel free to use subsections.  Take a look at the .tex file for this document and for the \LaTeX\ slide presentation for syntax.


But when there is a subsection 1, there must be at least a subsection 2 \ldots

\subsection{Lots on the web}

\vspace{0.2cm}
\noindent A good, {\bf free} text reference is 
``The Not So Short Introduction to \LaTeX2e'':

{\tt http://ctan.sharelatex.com/tex-archive/info/lshort/english/lshort.pdf}

\noindent Remember, {\em google is your friend}: anything you want to do in \LaTeX\ has probably been done and code can be found on the web.  

\section{Discussion and Conclusions}
\label{disc}

I like lists.
% if need to make an itemized list delineated by letters: must set up a new counter
\newcounter{a}
\begin{list}{\alph{a}.}{\usecounter{a}\setlength{\rightmargin}{\leftmargin}}
\item First item
\item Second item
\end{list}

% To add a figure; [h!] option tries to force LaTeX to place figure exactly in this spot of the document.

\begin{figure}[h!]
\begin{center}
\scalebox{.9}{\includegraphics{YokosukaJRSchedule.jpg}}
\end{center}
\caption{Japanese train schedules are often presented as stem-and-leaf plots!}
\label{JRsched}
\end{figure}


% Reference list on a new page since it does not count towards the page limit
\clearpage
\begin{thebibliography}{99}

\bibitem{9}
Robert, C. P.  (1995a).  Convergence Control Methods for Markov Chain
Monte Carlo Algorithms.  {\em Statistical Science} 10 (3), 231-253.

\bibitem{11}
Robert, C. P. and Casella, G.  (1999).  {\em Monte Carlo Statistical
Methods}.  Springer-Verlag, New York.

\end{thebibliography}

\clearpage
\noindent {\Large {\bf Appendix}}

\vspace{0.5cm}
The appendix does not count towards the page limit.  Put it at the end starting on a new page.  Make sure to place all R-code and output in the appendix.  The {\tt verbatim} environment will be a good one for this; here is an example of an R-code snippet in a verbatim environment.

\begin{verbatim}
# Load in required R packages
library(MASS)
library(corrplot)
library(car)
library(gam)

# Read in data on New York expenditures and clean a little by removing NAs
ny<-read.table("cs73.dat",header=T); dim(ny)  #  916  11
ny2<-na.omit(ny); dim(ny2) # 914  11
attach(ny2)
\end{verbatim}

\end{document}
